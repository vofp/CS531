\documentclass[11pt]{article}
\title{CS 531 Assignment\#2}
\author{
    Ryan Skeele\\
    Francis Vo
}
\date{}

\begin{document}
\maketitle
\section*{Intro}
Pseudo code of astar, rbfs
\section{Experiment Setup}
Nodes searched vs Problem size(astarh1, astarh2, rbfsh1, rbfsh2)
CPU vs Problem Size(...)
Table of solution lengths(...)
\section*{Discussion}
Show an example solution sequence for each algorithm for the largest size you tested in the following format:
%[0123,_,_], [123,0,_], [23,0,1], [23,_,01] ... 
Is there a clear preference ordering among the heuristics you tested considering the number of nodes searched and the total CPU time taken to solve the problems for the two algorithms?
Can a small sacrifice in optimality give a large reduction in the number of nodes expanded? What about CPU time?
How did you come up with your heuristic evaluation functions?
How do the two algorithms compare in the amount of search involved and the cpu-time?
Do you think that either of these algorithms scale to even larger problems? What is the largest problem you could solve with the best algorithm+heuristic combination? Report the wall-clock time, CPU-time, and the number of nodes searched.
Is there any tradeoff between how good a heuristic is in cutting down the number of nodes and how long it took to compute? Can you quantify it?
Is there anything else you found that is of interest?
\end{document}
